\section{Definizione della Sintassi }

Per specificare la sintassi di un linguaggio, si utilizza una notazione largamente impiegata nota come \textbf{grammatica libera dal contesto} (o semplicemente grammatica). Una grammatica descrive in modo naturale la struttura gerarchica della maggior parte dei costrutti dei linguaggi di programmazione.


\subsection{Definizione Formale di una Grammatica }
Una grammatica libera dal contesto è una quadrupla $G = (V, \Sigma, P, S)$ ed è definita da quattro componenti principali:
\begin{enumerate}
    \item Un insieme finito di simboli \textbf{terminali}, $\Sigma$, detti anche \textbf{token}. I terminali sono le unità lessicali elementari del linguaggio (es. parole chiave, operatori, cifre).
    
    \item Un insieme finito di simboli \textbf{non-terminali}, $V$, detti anche \textbf{variabili sintattiche}. Ogni non-terminale rappresenta un insieme di stringhe di terminali. Per definizione, gli insiemi $V$ e $\Sigma$ sono disgiunti, ovvero $V \cap \Sigma = \emptyset$.
    
    \item Un insieme finito di \textbf{produzioni}, $P$. Ogni produzione ha la forma $A \rightarrow \alpha$, dove:
    \begin{itemize}
        \item $A$ è un non-terminale ($A \in V$) e viene detta \textbf{testa} (o lato sinistro) della produzione.
        \item $\alpha$ è una sequenza (stringa) di terminali e/o non-terminali, $(\alpha \in (\Sigma \cup V)^*)$, e viene detta \textbf{corpo} (o lato destro) della produzione.
    \end{itemize}
    
    \item Un simbolo \textbf{iniziale} (o di partenza), $S$, scelto tra i non-terminali ($S \in V$).

\end{enumerate}
È detta libera dal contesto perché è possibile applicare una produzione indipendentemente da dove è A

\subsection{Derivazioni e Linguaggio Generato}
Una grammatica viene usata per generare (o \textbf{derivare}) le stringhe di un linguaggio.
\begin{itemize}
    \item \textbf{Derivazione diretta ($\Rightarrow$):} Data una stringa $\beta A \gamma$, dove $A$ è un non-terminale, se esiste una produzione $A \rightarrow \alpha$, possiamo sostituire $A$ con $\alpha$ per ottenere una nuova stringa. Scriviamo:
    \[ \beta A \gamma \Rightarrow \beta \alpha \gamma \]
   Questa operazione rappresenta un singolo passo di derivazione. 
    
    \item \textbf{Derivazione in zero o più passi ($\Rightarrow^*$):} Indica l'applicazione riflessiva e transitiva della derivazione diretta. Significa che una stringa $\alpha$ deriva una stringa $\gamma$ ($\alpha \Rightarrow^* \gamma$) se $\gamma$ può essere ottenuta da $\alpha$ tramite zero o più passi di derivazione. 
    
    \item \textbf{Derivazione in uno o più passi ($\Rightarrow^+$):} Simile a $\Rightarrow^*$, ma richiede almeno un passo di derivazione. 
    
    \item \textbf{Forma Sentenziale:} Si definisce \textbf{forma sentenziale} qualsiasi stringa $\beta$ che può essere derivata dal simbolo iniziale S in zero o più passi ($S \Rightarrow^* \beta$). 
    
    Una forma sentenziale è una "frase in costruzione": può contenere sia simboli \textbf{terminali} (le "parole" finali del linguaggio) sia simboli \textbf{non-terminali} (i "concetti" ancora da espandere).
    
    \item \textbf{Frase:} Una \textbf{frase} del linguaggio L(G) è una forma sentenziale composta \textit{unicamente} da simboli terminali. In altre parole, $w$ è una frase di G se e solo se $w \in \Sigma^*$ e $S \Rightarrow^+ w$.

    \item \textbf{Linguaggio Generato ($L(G)$):} Il linguaggio generato da una grammatica G è l'insieme di tutte le frasi che possono essere derivate dal simbolo iniziale. Formalmente:
    \[ L(G) = \{ w \in \Sigma^* \mid S \Rightarrow^+ w \} \]
  \end{itemize}
