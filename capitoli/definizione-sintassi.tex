\section{Definizione della Sintassi}

Per specificare la sintassi di un linguaggio, si utilizza una notazione largamente impiegata nota come \textbf{grammatica libera dal contesto} (o semplicemente grammatica). Una grammatica descrive in modo naturale la struttura gerarchica della maggior parte dei costrutti dei linguaggi di programmazione.

\subsection{Definizione Formale di una Grammatica}
Una grammatica libera dal contesto è una quadrupla $G = (V, \Sigma, P, S)$ ed è definita da quattro componenti principali:
\begin{enumerate}
    \item Un insieme finito di simboli \textbf{terminali}, $\Sigma$, detti anche \textbf{token}.
    \item Un insieme finito di simboli \textbf{non-terminali}, $V$. Per definizione, $V \cap \Sigma = \emptyset$.
    \item Un insieme finito di \textbf{produzioni}, $P$, nella forma $A \rightarrow \alpha$, dove $A \in V$ è la \textbf{testa} e $\alpha \in (\Sigma \cup V)^*$ è il \textbf{corpo}.
    \item Un simbolo \textbf{iniziale}, $S$, scelto tra i non-terminali ($S \in V$).
\end{enumerate}
Una grammatica è detta "libera dal contesto" perché è possibile applicare una produzione $A \rightarrow \alpha$ indipendentemente da dove si trova il non-terminale $A$.

\subsection{Definizioni Chiave}

\begin{itemize}
    \item \textbf{Derivazione Diretta ($\Rightarrow$):} 
    Rappresenta un singolo passo di riscrittura, dove un non-terminale viene sostituito dal corpo di una sua produzione. 
    \begin{itemize}
        \item Il passaggio da $E$ a $E + T$ è una \textbf{derivazione diretta}, poiché abbiamo applicato la produzione $E \rightarrow E + T$.
        \item Allo stesso modo, $T + T \Rightarrow \mathbf{id} + T$ è una derivazione diretta, dove il primo $T$ è stato sostituito usando la produzione $T \rightarrow \mathbf{id}$.
    \end{itemize}

    \item \textbf{Derivazione in zero o più passi ($\Rightarrow^*$):} 
    Indica una sequenza completa (o parziale) di derivazioni dirette. Poiché è possibile arrivare da $E$ a $\mathbf{id} + T$ in 3 passi, possiamo scrivere:
    \[ E \Rightarrow^* \mathbf{id} + T \]
    La relazione è riflessiva, quindi è corretto anche scrivere $E \Rightarrow^* E$ (zero passi).
    
    \item \textbf{Forma Sentenziale:} 
    È una qualsiasi stringa intermedia nel processo di derivazione che può contenere sia terminali che non-terminali. Nel nostro esempio, tutte le seguenti stringhe sono forme sentenziali:
    \begin{itemize}
        \item $E$ (la forma sentenziale iniziale)
        \item $E + T$
        \item $T + T$
        \item $\mathbf{id} + T$
    \end{itemize}

    \item \textbf{Frase:} 
    È una forma sentenziale che è composta \textit{unicamente} da simboli terminali. Rappresenta una stringa finale e valida del linguaggio.
    \begin{itemize}
        \item La stringa $\mathbf{id} + \mathbf{id}$ è una \textbf{frase}, perché è stata derivata dal simbolo iniziale e contiene solo i simboli terminali $\mathbf{id}$ e $+$.
        \item La stringa $\mathbf{id} + T$, invece, è una forma sentenziale ma \textbf{non} è una frase, perché contiene ancora il non-terminale $T$.
    \end{itemize}

    \item \textbf{Linguaggio Generato ($L(G)$):} 
    È l'insieme di tutte le frasi che una grammatica G può generare. La nostra frase $\mathbf{id} + \mathbf{id}$ è solo uno degli infiniti elementi del linguaggio $L(G)$ definito dalla grammatica d'esempio. Altre frasi sarebbero $\mathbf{id}$, $\mathbf{id} + \mathbf{id} + \mathbf{id}$, ecc.
    
\end{itemize}
\subsection{Esempio di Grammatica e Derivazione}
Consideriamo una grammatica per espressioni:
\begin{itemize}
    \item $V = \{E, I\}$
    \item $\Sigma = \{a, b, 0, 1, +, *, (, )\}$
    \item $S = E$
    \item $P = \{ E \rightarrow I \mid E+E \mid E*E \mid (E) \, ; \quad I \rightarrow a \mid b \mid Ia \mid Ib \mid I0 \mid I1 \}$
\end{itemize}

Verifichiamo che la stringa \texttt{ab*(b01+ba)} appartenga al linguaggio generato:
\begin{align*}
E & \Rightarrow E * E \\
  & \Rightarrow I * E \\
  & \Rightarrow Ia * E \Rightarrow ab * E \\
  & \Rightarrow ab * (E) \\
  & \Rightarrow ab * (E + E) \\
  & \Rightarrow ab * (I + E) \\
  & \Rightarrow ab * (I0 + E) \Rightarrow ab * (I01 + E) \Rightarrow ab * (b01 + E) \\
  & \Rightarrow ab * (b01 + I) \\
  & \Rightarrow ab * (b01 + Ia) \Rightarrow ab * (b01 + ba)
\end{align*}

\subsection{Albero di Parsing (Parse Tree)}
Un albero di parsing rappresenta graficamente il modo in cui una stringa del linguaggio
può essere derivata dal simbolo iniziale.
\begin{itemize}
    \item \textbf{Struttura:} La radice è il simbolo iniziale. I nodi interni sono non-terminali. Le foglie sono terminali e, lette da sinistra a destra, formano la frase derivata.
    \item \textbf{Visita:} La frase originale viene fuori facendo una visita in pre-ordine dell'albero, esaminando da sinistra la zona più "profonda".
\end{itemize}
\vspace{0.3 cm}
% --- Rappresentazione testuale dell'albero di parsing ---
\begin{Verbatim}[frame=single, label=Albero di Parsing per ab*(b01+ba)]
                E
                |
        ------------------
       |        |         |
       E        * E
       |                  |
       I                 (E)
       |                  |
   ---------          ---------
  |         |        |    |    |
  I         b        E    +    E
  |                    |         |
  a                    I         I
                       |         |
                   ---------   -----
                  |    |    |   |   |
                  I    0    1   I   a
                  |             |
                  b             b
\end{Verbatim}


\subsection{Correttezza e Completezza}
Una grammatica G è corretta e completa rispetto a un linguaggio L se $L(G) = L$.
\begin{itemize}
    \item \textbf{Correttezza:} Ogni stringa derivabile appartiene al linguaggio ($S \Rightarrow^* w \implies w \in L$).
    \item \textbf{Completezza:} Ogni stringa del linguaggio è derivabile ($w \in L \implies S \Rightarrow^* w$).
\end{itemize}

\vspace{0.2 cm}
\textbf{Dimostrazione di Completezza (per induzione su $|w|$):}
\begin{itemize}
    \item \textbf{Base:} Se $|w|=0$, allora $w=\epsilon$. Se $\epsilon \in L(G)$, allora deve esistere $S \Rightarrow^* \epsilon$.
    \item \textbf{Passo Induttivo:} Supponiamo che per ogni stringa $w$ con $|w| \le n$ valga la tesi. Se $|w|=n+1$, allora $w$ sarà nella forma $u \text{ op } v$ o simile, e per ipotesi induttiva esisteranno le derivazioni per $u$ e $v$.
\end{itemize}

\vspace{0.1 cm}
\textbf{Dimostrazione di Correttezza (induzione sul numero di passi della derivazione):}
\begin{itemize}
    \item \textbf{Base:} Se la derivazione ha 1 passo ($S \Rightarrow^1 w$), si dimostra che $w \in L$. Ad esempio, se le produzioni con un passo sono $S \rightarrow \epsilon \mid 0 \mid 1$, allora le stringhe $\epsilon, 0, 1$ appartengono al linguaggio.
    
    \item \textbf{Passo Induttivo:} Si assume che per ogni derivazione di lunghezza $\le n$ la tesi sia valida. Per una derivazione di $n+1$ passi, si dimostra che anche la stringa risultante appartiene al linguaggio, basandosi sul fatto che è ottenuta da una stringa derivabile in $n$ passi. \\
    Esempio: $S \Rightarrow 0S0 \Rightarrow^* 0x0$. Se per ipotesi induttiva $x \in L$, e la regola di costruzione del linguaggio prevede che $0x0$ sia una stringa valida, allora la tesi è dimostrata.
\end{itemize}